\documentclass{paper}

\usepackage{epsfig}
\usepackage{graphicx}
\usepackage{amsmath}
\usepackage{amssymb}
\usepackage{color}
\usepackage{caption}
\usepackage{subcaption}
\usepackage{float}
\usepackage{bbm}



% something NOT relevant to the usage of the package.
\setlength{\parindent}{0pt}
\setlength{\parskip}{18pt}

\usepackage[latin1]{inputenc} 
\usepackage[T1]{fontenc} 

\usepackage{listings} 
\lstset{% 
   language=Matlab, 
   basicstyle=\small\ttfamily, 
} 

\title{Assignment 1}

\author{Simon Jenni\\09-116-005}
% //////////////////////////////////////////////////

\begin{document}



\maketitle


% Add figures:
%\begin{figure}[t]
%%\begin{center}
%\quad\quad   \includegraphics[width=1\linewidth]{ass2}
%%\end{center}
%
%\label{fig:performance}
%\end{figure}

\section*{P1}

\begin{enumerate}
\item  
Let $A_{SQ}$ be the area of the unit square and $A_{QC}$ be the area of the quarter-circle. The probability of a uniformly chosen point in $A_{SQ}$ lying in $A_{QC}$ is then given by:
\begin{equation}
P(x\in A_{QC}) = \frac{A_{QC}}{A_{SQ}} = \frac{\pi}{4}
\end{equation}

\item 
See attached Matlab files (without for-loops)
\item
See attached Matlab files (without for-loops)

\item
Running the program gives the following solutions:

\begin{table}[!htbp]
\centering
\begin{tabular}{|l|l|l|l|l|}
\hline
N        & $\pi_{min}$ & $\pi_{max}$ & $\pi_{mean}$ & $\Delta \pi$ \\ \hline
100      & 2.60000     & 3.48000     & 3.14640      & 0.17077      \\ \hline
1000     & 3.03600     & 3.24000     & 3.14304      & 0.04473      \\ \hline
10000    & 3.10280     & 3.18200     & 3.14028      & 0.01555      \\ \hline
100000   & 3.13092     & 3.15756     & 3.14226      & 0.00500      \\ \hline
1000000  & 3.13752     & 3.14668     & 3.14169      & 0.00167      \\ \hline
10000000 & 3.14047     & 3.14283     & 3.14164      & 0.00052      \\ \hline
\end{tabular}
\end{table}
\end{enumerate}


\section*{P2}
\begin{enumerate}
\item 
Let $X_i\in{\{"yes", "no"\}}$ for $i=1,...,N$.
The rule could be as follows:
\begin{equation}
\begin{cases}
    "yes"& \text{if } \sum_{i=1}^{N} \mathbbm{1}\{X_i="yes")\} \geq N/2 \\
    "no",              & \text{otherwise}
\end{cases}
\end{equation}
That is, we would choose $"yes"$ if we encounter $"yes"$ at least five times in all the $X_i$. In the particular case we would therefore choose "yes" as it occurs six times.

\item
Yes, the rule would have to be modified as the probability of error in one of the two cases would be greater than the other one. Weighting the outcomes by the inverse of the error probabilities sounds like a good idea.



\end{enumerate}

 \end{document}
 
 