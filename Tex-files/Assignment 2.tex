\documentclass{paper}

\usepackage{epsfig}
\usepackage{graphicx}
\usepackage{amsmath}
\usepackage{amssymb}
\usepackage{color}
\usepackage{caption}
\usepackage{subcaption}
\usepackage{float}
\usepackage{bbm}



% something NOT relevant to the usage of the package.
\setlength{\parindent}{0pt}
\setlength{\parskip}{18pt}

\usepackage[latin1]{inputenc} 
\usepackage[T1]{fontenc} 

\usepackage{listings} 
\lstset{% 
   language=Matlab, 
   basicstyle=\small\ttfamily, 
} 

\title{Assignment 2}

\author{Simon Jenni\\09-116-005}
% //////////////////////////////////////////////////

\begin{document}



\maketitle


% Add figures:
%\begin{figure}[t]
%%\begin{center}
%\quad\quad   \includegraphics[width=1\linewidth]{ass2}
%%\end{center}
%
%\label{fig:performance}
%\end{figure}

\section*{P1 and P2}
Please see attached MATLAB files...

\section*{P3}
We can write each number $x \in{(0,1)}$ as a possibly infinite series 
\begin{equation}
x = \sum_{i=1}^{\infty}b_i*2^{-i} 
\end{equation}
where $b_i\in{\{0, 1\}}$ and $b = (b_1, b_2, ...)$ is a binary representation of $x$. 

We can then use the random number generator to create the $b_i$ via the mapping 
\begin{equation}
b_i = 
\begin{cases}
    1& \text{if result is } a \\
    0              & \text{otherwise}
\end{cases}
\end{equation}

This will result in uniformly distributed random numbers in $(0,1)$. However, note that infinitely many samples of the RNG would have to be produced in order to generate the infinitely many $b_i$'s.

 \end{document}
 
 